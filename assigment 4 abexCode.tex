\documentclass{article}
\usepackage[utf8]{inputenc}

\title{reCAPTCHA}
\author{ojiamboabex }
\date{March 2018}

\begin{document}

\maketitle

\section{reCAPTCHA}
\section{What is reCAPTCHA?}

reCAPTCHA is a free service that protects your site from spam and abuse. It uses advanced risk analysis techniques to tell humans and bots apart. With the new API, a significant number of your valid human users will pass the reCAPTCHA challenge without having to solve a CAPTCHA. reCAPTCHA comes in the form of a widget that you can easily add to your blog, forum, registration form, etc.
\section{Developer's Guide}
reCAPTCHA protects you against spam and other types of automated abuse. Here, we explain how to add reCAPTCHA to your site or application.
\newline This documentation is designed for people familiar with HTML forms, server-side processing or mobile application development. To install reCAPTCHA, you will probably need to edit some code.
\section{Overview}
To start using reCAPTCHA, you need to sign up for an API key pair for your site. The key pair consists of a site key and secret key. The site key is used to invoke reCAPTCHA service on your site or mobile application. The secret key authorizes communication between your application backend and the reCAPTCHA server to verify the user's response. The secret key needs to be kept safe for security purposes.
\newline First, choose the type of reCAPTCHA and then fill in authorized domains or package names. After you accept our terms of service, you can click Register button to get new API key pair.
Now please take the following steps to add reCAPTCHA to your site or mobile application:
\newline 1.	Choose the client side integration:
\newline a.	reCAPTCHA V2
\newline b.	Invisible reCAPTCHA
\newline c.	reCAPTCHA Android Library
\newline 2.	Verifying the user's response

\section{Some asked questions}
What should I do to run automated tests with reCAPTCHA v2.?
\newline Can I customize the reCAPTCHA widget?
\section{References}
 [1]"reCAPTCHA  |  Google Developers", Google Developers, 2018. [Online]. Available: https://developers.google.com/recaptcha/. [Accessed: 09- Mar- 2018].



\end{document}
